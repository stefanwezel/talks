\documentclass{article} % For LaTeX2e
\usepackage{neurips,times}
\usepackage{hyperref}
\usepackage{url}
\usepackage{booktabs}       % professional-quality tables
\usepackage{graphicx}
\usepackage{lipsum}
\usepackage{amsmath}
\usepackage{multirow}
\usepackage{siunitx}
\usepackage{pgfplots}
\pgfplotsset{compat=newest}

\usepackage[utf8]{inputenc} % allow utf-8 input
\usepackage[T1]{fontenc}    % use 8-bit T1 fonts
\usepackage{hyperref}       % hyperlinks
\usepackage{url}            % simple URL typesetting
\usepackage{booktabs}       % professional-quality tables
\usepackage{amsfonts}       % blackboard math symbols
\usepackage{nicefrac}       % compact symbols for 1/2, etc.
\usepackage{microtype}      % microtypography
\usepackage{amssymb,amsmath,bm}
\usepackage{color,soul}
\usepackage{multirow}
\usepackage{mathtools}
\usepgfplotslibrary{groupplots}


\def\figwidth{.5\linewidth}
\def\figheight{.15\textheight}
%\newlength{\figwidth}{\mywidth}
%\newlength{\figheight}{.2\textheight}


\usepackage[numbers]{natbib}
\setlength{\bibsep}{0.0pt}

\title{Predicting the Solar Potential of Rooftops using
	Image Segmentation and Structured Data\\ \vspace{0.5cm}\large{Report}}
\author{Stefan Wezel \\ stefan.wezel@student.uni-tuebingen.de \\4080589  \\ ML4S}

\newcommand{\fix}{\marginpar{FIX}}
\newcommand{\new}{\marginpar{NEW}}

\nipsfinalcopy % for line numbers

\begin{document}
%\setlength{\figwidth}{.8\textwidth}
%\setlength{\figheight}{.2\textheight}
\maketitle

\begin{abstract}
	Solar panels are a cost effective solution for generating energy in a carbon-free manner. However, not every roof is suitable for installing solar panel. Architecture and location heavily effect the viability of such systems.
	Predicting this solar potential of a roof is traditionally a labour intensive process requiring on site measurements. Automating this process and scale it up is a difficult challenge. Here, we will introduce a solution proposed by \citet{de2021predicting}, review it, and compare it to other approaches.
\end{abstract}

\section*{Introduction}
%- addressed problem: solar potential prediction
%- how does paper approach the problem: hybrid method of ml and analytic solutions incorporating many different data sources (very practical)
%- How do other papers solve the problem or was it not possible before:
%- Could not be solved manually before, because scale was simply too large
%- other methods: google, ...
%- Cons of method: evaluation is difficult
%- other ways to address the problem with ml: -> end2end?
%- core message: incorporating knowledge and ml in a complex system to solve a real world problem
In the European Union (EU) alone, rooftops make up an estimate area of \SI{7935}{\kilo\metre\squared}~\cite{bodis2019high}. Much of this area could be used to install solar panels and help feed demand for renewably generated energy. Predicting how much energy a roof could produce once panels are installed. This is referred to as a roofs solar potential and is a crucial task. Locally, to determine the viability and economic efficiency of solar panels. Globally, it could also help producing a guess of how much solar energy could contribute to overall energy production capabilities.\\
Traditionally, a roofs solar potential is estimated by performing measurements of roof geometry, considering its geographic location, and architecture of surrounding buildings or vegetation \cite{freitas2015modelling}. While more recently, geographic information systems (GIS) play an increasingly large role in guiding solar development, much of the process is still labour and time consuming. Thus, solar potential estimation on a large scale remains challenging.\\
Machine learning offers promising capabilities to increase the magnitude on which solar potential estimation can be performed. However, due to limited and complex data it is not a trivial problem. A solution is proposed by \citet{de2021predicting}. They incorporate structured data and existing knowledge as inductive bias to a method that combines machine learning and analytical methods.

\section*{Background \& Existing approaches}
\citet{freitas2015modelling} present an overview of approaches combining algorithms and GIS modeling to estimate solar potential in dense urban environments. They compare different numeric solar radiation algorithms and data sources ranging from 2d maps to high resolution 3d models of urban scenes. In their survey, they find that major factors limiting these approaches include poor data quality and the difficulty of validating models.\\
\cite{bodis2019high} use high-resolution satellite data and statistical information to produce an estimate of solar potential across the whole EU. They also include economical calculations in their method to estimate viability of installing solar. However, their method only yields estimates for areas and not for specific rooftops.\\
A similar approach is proposed by \citet{ouammi2012artificial} who use rudimentary neural network architectures and focus on Moroccan territory. \citet{assouline2018large} focus on Switzerland and use random forests for their predictions.\\
With project sunroof, the technology company Google has proposed an approach that offers fine-grained solar potential estimation for individual rooftops within the United States and Puerto Rico. From Google Maps data, they find rooftop outlines using a (not further specified) deep learning method. They then estimate rooftop geometry and use historical weather data to predict the solar potential \cite{sunroof}.\\
Further private sector endeavors include a cooperation between the companies Otovo and In Sun We Trust that offer a product similar to Project Sunroof but only serve France \cite{insun}. Other existing products focus on small areas or only offer solar potential estimates on-demand \cite{insun, rhino}.\\
\citet{lee2019deeproof} propose a data-driven method that mostly relies on widely available satellite data. They estimate roof topology directly from this imagery using image segmentation architectures. They then use further public data of solar radiance to estimate solar potential. They validate their method by comparing it to a precise but expensive LIDAR-based approach. Their method can be applied in a wide variety of settings.\\
Several other approaches leveraging sophisticated deep learning methods are proposed for solar potential estimation \cite{huang2019urban, zhong2021city} or solar irradiance mapping \cite{kumari2021deep, chandola2020multi, bamisile2020application}.




\section*{Method}

\section*{Discussion}


\section*{Conclusion}


\newpage

\bibliographystyle{unsrtnat}
\bibliography{refs}

\end{document}
