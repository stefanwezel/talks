% !TeX program = xelatex
\documentclass[9pt]{beamer}
\usepackage{xcolor}
\definecolor{orange}{HTML}{EF767A}
\definecolor{lightgray}{HTML}{CCCCCC}
\definecolor{red}{HTML}{AC454A}
\definecolor{brown}{HTML}{EAD296}
\definecolor{darkgrey}{HTML}{313630}
\definecolor{cornflower}{HTML}{247BA0}
\definecolor{sienna}{HTML}{6C464F}
\usefonttheme{professionalfonts} % using non standard fonts for beamer
\usefonttheme{serif} % default family is serif
\usepackage{fontspec}
\usepackage{setspace}
\usepackage{natbib}
\usepackage{animate}
\usepackage{graphicx}
%\usepackage[T1]{fontenc}

\bibliographystyle{abbrv}
%\setmainfont{Liberation Serif}
%\setmainfont{Liberation Serif}
\setmainfont{Comfortaa}
%\usepackage[T1]{fontenc}

\setbeamercolor{frametitle}{bg=orange,fg=white}
\setbeamercolor{author in head/foot}{bg=orange,fg=white}

%\setbeamerfont{page number}{size=\Huge}

%\setbeamertemplate{itemize itemjpegs}[circle]
\useinnertheme{circles}
\setbeamercolor{palette primary}{bg=orange,fg=white}
%\setbeamercolor{palette secondary}{bg=red,fg=white}
\setbeamertemplate{itemize item}{\color{darkgrey}$\circ$}
\setbeamercolor{structure}{fg=red} % itemize, enumerate, etc

%\setbeamercolor{section in head/foot}{bg=red}
\setbeamercolor{title}{fg=orange} %, bg=brown
\setbeamercolor{author}{fg=darkgrey}
\setbeamercolor{institute}{fg=darkgrey}
\setbeamercolor{date}{fg=darkgrey}
%\setbeamercolor{normal text}{fg=darkgrey}
\makeatletter
\setbeamertemplate{headline}{%
	\usebeamercolor[bg]{frametitle}\rule{\textwidth}{1cm}
}
\setbeamerfont{title}{size=\LARGE}
\setbeamerfont{institute}{size=\normalsize}
\renewcommand*{\bibfont}{\scriptsize}


\setbeamertemplate{frametitle}{%
	\vskip-1cm%
	\begin{minipage}[c][\headheight][c]{\textwidth}%
		\usebeamerfont{frametitle}
		\strut\insertframetitle\par
		{%
			\ifx\insertframesubtitle\@empty%
			\else%
			{\usebeamerfont{framesubtitle}\usebeamercolor[fg]{framesubtitle}\strut\insertframesubtitle\par}%
			\fi
		}%      
		\vspace*{0.05cm}
	\end{minipage}%
	\vskip-0.1em
}
%\setbeamertemplate{footline}{%
%	\leavevmode%
%	\hbox{\begin{beamercolorbox}[wd=\paperwidth,ht=4.5ex,dp=3.125ex]{author in head/foot}%
%			\usebeamerfont{author in head/foot} bar
%	\end{beamercolorbox}}%
%	\vskip0pt%
%}
\makeatother


\title{Unsupervised Learning \\
	of Disentangled and Interpretable Representations from Sequential Data}
\author{Wei-Ning Hsu, Yu Zhang, and James Glass\\Talk by Stefan Wezel}
\institute{Seminar ML4S}

\date{\today}


%\setbeamertemplate{sidebar right}{}
%\setbeamertemplate{footline}{%
%	\hfill\usebeamertemplate***{navigation symbols}
%	\hspace{1cm}\insertframenumber{}}
\setbeamerfont{page number in head/foot}{size=\small}
    \setbeamertemplate{footline}{%
	\raisebox{5pt}{\makebox[\paperwidth]{\hfill\makebox[10pt]{\scriptsize\insertframenumber}}}}
\setbeamertemplate{navigation symbols}{}
%\onehalfspacing
\setstretch{1.3}
\begin{document}
	

\setbeamercolor{background canvas}{bg=white}
\setbeamercolor{normal text}{fg=darkgrey}
\usebeamercolor[fg]{normal text}
\begin{frame}[plain]
	\titlepage
\end{frame} 



\setbeamercolor{background canvas}{bg=white}
\setbeamercolor{normal text}{fg=darkgrey}
\usebeamercolor[fg]{normal text}
\setbeamertemplate{itemize item}{\color{darkgrey}$\circ$}
\begin{frame}
\frametitle{Overview}
%\framesubtitle{}
\begin{itemize}%\setlength\itemsep{1.5em}
	\item Introduction
	\item What are Disentangled Representations? (intuition)
	\item Why Disentangled Representations
	\item Formal Description of Disentangled Representations
	\item Disentanglement in the Context of the Paper
	\item The Factorized Hierarchical VAE Model
	\item Results
	%\item Other approaches and challenges
\end{itemize}
\end{frame} 



\setbeamercolor{background canvas}{bg=white}
\setbeamercolor{normal text}{fg=darkgrey}
\usebeamercolor[fg]{normal text}
\setbeamertemplate{itemize item}{\color{darkgrey}$\circ$}
\begin{frame}
\frametitle{Introduction}
\framesubtitle{Setting - Speech data}
\begin{itemize}%\setlength\itemsep{1.5em}
	\item Propose Sequential Factorized Hierarchical VAE (\textbf{FHVAE})
	%\item Learn factorized latent space
	\item Focus on speech data
	\begin{itemize}
		\item Sequence level (Speaker, ...) attributes
		\item Segment level (content, noise, ...) attributes
	\end{itemize}
	\item Reflect different temporal scales in latent space
\end{itemize}
\end{frame} 




\setbeamercolor{background canvas}{bg=white}
\setbeamercolor{normal text}{fg=darkgrey}
\usebeamercolor[fg]{normal text}
\setbeamertemplate{itemize item}{\color{darkgrey}$\circ$}
\begin{frame}
\frametitle{What is disentanglement?}
\framesubtitle{Intuition}
	\begin{itemize}%\setlength\itemsep{1.5em}
	\item Encode distinct generating factors in separate subsets of latent space dimensions
	%\item i.e. color as one subspace, translation, as another
	%\item The exact definition is often discussed, we will have a look at a proposed one
	\end{itemize}
%\animategraphics[autoplay,loop,width=\linewidth/2]{10}{disentangled_scene-}{0}{19}
\begin{figure}
	\includegraphics[width=.93\linewidth]{figures/intution_3x3_1.pdf}
	%\caption{World states are determinded by two generative factors, a change in those is reflected in the two corresponding latent variables, our $f$ maps to}
\end{figure}
\end{frame}
\setbeamercolor{background canvas}{bg=white}
\setbeamercolor{normal text}{fg=darkgrey}
\usebeamercolor[fg]{normal text}
\setbeamertemplate{itemize item}{\color{darkgrey}$\circ$}
\begin{frame}
\frametitle{What is disentanglement?}
\framesubtitle{Intuition}
\begin{itemize}%\setlength\itemsep{1.5em}
	\item Encode distinct generating factors in separate subsets of latent space dimensions
	%\item i.e. color as one subspace, translation, as another
	%\item The exact definition is often discussed, we will have a look at a proposed one
\end{itemize}
%\animategraphics[autoplay,loop,width=\linewidth/2]{10}{disentangled_scene-}{0}{19}
\begin{figure}
	\includegraphics[width=.93\linewidth]{figures/intution_3x3_2.pdf}
	%\caption{World states are determinded by two generative factors, a change in those is reflected in the two corresponding latent variables, our $f$ maps to}
\end{figure}
\end{frame} 
\setbeamercolor{background canvas}{bg=white}
\setbeamercolor{normal text}{fg=darkgrey}
\usebeamercolor[fg]{normal text}
\setbeamertemplate{itemize item}{\color{darkgrey}$\circ$}
\begin{frame}
\frametitle{What is disentanglement?}
\framesubtitle{Intuition}
\begin{itemize}%\setlength\itemsep{1.5em}
	\item Encode distinct generating factors in separate subsets of latent space dimensions
	%\item i.e. color as one subspace, translation, as another
	%\item The exact definition is often discussed, we will have a look at a proposed one
\end{itemize}
%\animategraphics[autoplay,loop,width=\linewidth/2]{10}{disentangled_scene-}{0}{19}
\begin{figure}
	\includegraphics[width=.93\linewidth]{figures/intution_3x3_3.pdf}
	%\caption{World states are determinded by two generative factors, a change in those is reflected in the two corresponding latent variables, our $f$ maps to}
\end{figure}
\end{frame} 





\setbeamercolor{background canvas}{bg=white}
\setbeamercolor{normal text}{fg=darkgrey}
\usebeamercolor[fg]{normal text}
\setbeamertemplate{itemize item}{\color{darkgrey}$\circ$}
\begin{frame}
\frametitle{Why learn disentangled representations?}
\framesubtitle{Motivation}
\begin{itemize}%\setlength\itemsep{1.5em}
	\item Explainability/Interpretability
	\item Fairness
	\item Scientific modeling %we often know, what generating factors are, and want our model to reflect those
	\item Speaker verification %independent of content
	\item Denoising
\end{itemize}
\end{frame} 



\setbeamercolor{background canvas}{bg=white}
\setbeamercolor{normal text}{fg=darkgrey}
\usebeamercolor[fg]{normal text}
\setbeamertemplate{itemize item}{\color{darkgrey}$\circ$}
\begin{frame}
\frametitle{Disentangled Representations Formally}
\framesubtitle{A field-trip to group theory: important concepts}
\begin{itemize}%look at simple example here
	\item Group
	\begin{itemize}
		\item Operation and non-empty set $G=(\circ, G)$
		\item Set closed under operation, identity element, inverse elements, associativity
	\end{itemize}
	\item Symmetry group
	\begin{itemize}
		\item Set of transformations that leave object $X$ invariant %(i.e. another set)
		\item Operation is composition of transformations
	\end{itemize}
	\item Group action
	\begin{itemize}
		\item Results of symmetry transformations on object $X$
		\item I.e. set of changed order %(permuation)
		%For example permutation
	\end{itemize}
	\item Direct product
	\begin{itemize}
		\item $G = G_1 \times ... \times G_n$
		%\item Group conditions must hold for group and each subgroup
	\end{itemize}
 \end{itemize}
\end{frame} 



\setbeamercolor{background canvas}{bg=white}
\setbeamercolor{normal text}{fg=darkgrey}
\usebeamercolor[fg]{normal text}
\setbeamertemplate{itemize item}{\color{darkgrey}$\circ$}
\begin{frame}
\frametitle{Disentangled Representations Formally}
\framesubtitle{A field-trip to group theory: What is disentanglement in terms of group theory?}
\begin{itemize}%\setlength\itemsep{1.5em}
	\item Disentangled group actions
	\begin{itemize}
		\item Result of subset of symmetries $G_i$ that only change subset $X_i$ of object, but leave other $X_{j \neq i}$ invariant
	\end{itemize}
	\item $\rightarrow$ If we observe disentangled group actions in the world, we want to model those
	%G are all the symmetries of our beautiful world
	\item We can assume $G$ can be decomposed into direct product of symmetry subgroups $G_i$
\end{itemize}
\end{frame} 

\setbeamercolor{background canvas}{bg=white}
\setbeamercolor{normal text}{fg=darkgrey}
\usebeamercolor[fg]{normal text}
\setbeamertemplate{itemize item}{\color{darkgrey}$\circ$}
\begin{frame}
\frametitle{Disentangled Representations Formally}
\framesubtitle{A field-trip to group theory: What is disentanglement in terms of group theory?}
\begin{itemize}%\setlength\itemsep{1.5em}
	\item We want to find symmetry preserving mapping $f:X \mapsto Z$
	%\item Symmetry $G$ on $X$ should be preserved in $Z$%, $G \times Z \mapsto Z$
%	\begin{itemize}
%		\item $g \cdot f(x) = f(g\cdot x)$ $\rightarrow$ equivariant map
%	\end{itemize}
	\begin{align*}
		&X \;\xrightarrow[\text{}]{G}\;\;X\\ 
		f&\downarrow \;\;\;\;\;\;\;f\downarrow\\
		&Z \;\xrightarrow[\text{}]{G}\;\;\;Z
	\end{align*}
	\item $\rightarrow$ Equivariant map $g \cdot f(x) = f(g\cdot x)$
	\item Result is disentangled representation
	\begin{itemize}
		%representation reflects transformation applied to input
		\item Decomposition $Z = Z_{1} \times ... \times Z_{n}$
		\item $Z_{i}$ only affected by symmetry $G_i$ on $X$% ($G_{j\neq i}$)
		\item $Z_i$ invariant to all $G_{j \neq i}$% on $X$ %$Z_{warped}$ is only affected by warps in $W$ ($G_{warps}$) % G acts on W this affects Z
		%\item $\rightarrow$ Each subspace $Z_i$ can be transformed ONLY by the corresponding symmetry $G_i$% on $W$ (or on $Z$)%(like shift or warp independently)
		%\item We can say, $G_i$ results in disentangled group actions in $Z$
	\end{itemize}
\end{itemize}
\end{frame} 

%\setbeamercolor{background canvas}{bg=white}
%\setbeamercolor{normal text}{fg=darkgrey}
%\usebeamercolor[fg]{normal text}
%\setbeamertemplate{itemize item}{\color{darkgrey}$\circ$}
%\begin{frame}
%\frametitle{Disentangled Representations Formally}
%\framesubtitle{A field-trip to group theory: What is disentanglement in terms of group theory?}
%\begin{itemize}%\setlength\itemsep{1.5em}
%	\item Representation is disentangled if:
		%\item action $\cdot : G \times Z \mapsto Z$
%		\item equivariant map $f:X \mapsto Z$%, g \cdot f(w) = f(g\cdot w) \forall g \in G,w \in W$ % %between actions on $W$ and $Z$
%		\item such a map would split $Z$ into independent subspaces, thus satisfying:
%		\begin{itemize}
			%representation reflects transformation applied to input
%			\item Decomposition $Z = Z_{1} \times ... \times Z_{n}$
%			\item where $Z_{i}$ is only affected by transformations $G_i$ in $W$% ($G_{j\neq i}$)
%			\item and $Z_i$ invariant to all $G_{j \neq i}$ in $W$ %$Z_{warped}$ is only affected by warps in $W$ ($G_{warps}$) % G acts on W this affects Z
%			\item Thus each subspace $Z_i$ can be transformed ONLY by the corresponding symmetry $G_i$ on $W$ (or on $Z$)%(like shift or warp independently)
			%\item We can say, $G_i$ results in disentangled group actions in $Z$
%		\end{itemize}
		%\item action $\cdot$ on $Z$ is disentangled according to definition from before
	%\item There may be more criteria (preserving group structure, isomorphisms, ...) but for the intuition this is sufficient
	%\item (Note that much prior knowledge about our world was injected here)
%\end{itemize}
%\end{frame} 



%\setbeamercolor{background canvas}{bg=white}
%\setbeamercolor{normal text}{fg=darkgrey}
%\usebeamercolor[fg]{normal text}
%\setbeamertemplate{itemize item}{\color{darkgrey}$\circ$}
%\begin{frame}
%\frametitle{Disentangled Representations Formally}
%\framesubtitle{A field-trip to group theory: Disentangle our example formally}
%\begin{itemize}%\setlength\itemsep{1.5em}
%	\item signal $x(t) = sin(h_0 \cdot t) + sin(h_1 \cdot t)$ with $h_0 \sim \mathcal{N}(0,1),\;h_1 \sim \mathcal{N}(5,1);$
%	\item The set of possible values for $h_0, h_1$ make up our $W$
%	\item The group of symmetries acting on this $W$ decompose into $G = G_{h_0} \times G_{h_1}$
%	\item We want to find an equivariant map $f:W\mapsto Z$ with $Z \in \mathbb{Z}^2$
%	\item so that changes of $h_0$ result ONLY in changes in $z_0$ and changes of $h_1$ ONLY in $z_2$
%	\item Note, that this requires prior knowledge of generating factors in our world
%\end{itemize}
%\end{frame} 





\setbeamercolor{background canvas}{bg=white}
\setbeamercolor{normal text}{fg=darkgrey}
\usebeamercolor[fg]{normal text}
\setbeamertemplate{itemize item}{\color{darkgrey}$\circ$}
\begin{frame}
\frametitle{Back to the paper}
\framesubtitle{Did they achieve disentanglement?}
\begin{itemize}
	\item Disentangled with respect to what decomposition?
	\item Assume decomposition $G = G_{sequence} \times G_{segment}$
	\item Reflect decomposition in $Z= (z_1, z_2)$
	\item $z_1$: segment, $z_2$: sequence
	%\item They propose to store sequence information in $z_2$ and segment information in $z_1$
	\item Find equivariant map $f:W\mapsto Z$, so that $z_2$ is only affected by symmetries $G_{sequence}$ and vice versa%actions of $G_{sequence}$ 
\end{itemize}
\begin{figure}
	\includegraphics[width=.9\linewidth]{figures/fm_variables.png}
\end{figure}
\begin{tiny}
	Image source: Hsu et al., 2017 
\end{tiny}
\end{frame} 



\setbeamercolor{background canvas}{bg=white}
\setbeamercolor{normal text}{fg=darkgrey}
\usebeamercolor[fg]{normal text}
\setbeamertemplate{itemize item}{\color{darkgrey}$\circ$}
\begin{frame}
\frametitle{FHVAE}
\framesubtitle{Encoder$_{\theta}$}
\begin{figure}
	\includegraphics[width=1.05\linewidth]{figures/fhvae_decoder_0.pdf}
\end{figure}
\end{frame}
\setbeamercolor{background canvas}{bg=white}
\setbeamercolor{normal text}{fg=darkgrey}
\usebeamercolor[fg]{normal text}
\setbeamertemplate{itemize item}{\color{darkgrey}$\circ$}
\begin{frame}
\frametitle{FHVAE}
\framesubtitle{Encoder$_{\theta}$}
\begin{figure}
	\includegraphics[width=1.05\linewidth]{figures/fhvae_decoder_1.pdf}
\end{figure}
\end{frame} 
\setbeamercolor{background canvas}{bg=white}
\setbeamercolor{normal text}{fg=darkgrey}
\usebeamercolor[fg]{normal text}
\setbeamertemplate{itemize item}{\color{darkgrey}$\circ$}
\begin{frame}
\frametitle{FHVAE}
\framesubtitle{Encoder$_{\theta}$}
\begin{figure}
	\includegraphics[width=1.05\linewidth]{figures/fhvae_decoder_2.pdf}
\end{figure}
\end{frame} 
\setbeamercolor{background canvas}{bg=white}
\setbeamercolor{normal text}{fg=darkgrey}
\usebeamercolor[fg]{normal text}
\setbeamertemplate{itemize item}{\color{darkgrey}$\circ$}
\begin{frame}
\frametitle{FHVAE}
\framesubtitle{Encoder$_{\theta}$}
\begin{figure}
	\includegraphics[width=1.05\linewidth]{figures/fhvae_decoder_3.pdf}
\end{figure}
\end{frame} 
\setbeamercolor{background canvas}{bg=white}
\setbeamercolor{normal text}{fg=darkgrey}
\usebeamercolor[fg]{normal text}
\setbeamertemplate{itemize item}{\color{darkgrey}$\circ$}
\begin{frame}
\frametitle{FHVAE}
\framesubtitle{Encoder$_{\theta}$}
\begin{figure}
	\includegraphics[width=1.05\linewidth]{figures/fhvae_decoder_4.pdf}
\end{figure}
\end{frame} 
\setbeamercolor{background canvas}{bg=white}
\setbeamercolor{normal text}{fg=darkgrey}
\usebeamercolor[fg]{normal text}
\setbeamertemplate{itemize item}{\color{darkgrey}$\circ$}
\begin{frame}
\frametitle{FHVAE}
\framesubtitle{Decoder$_{\phi}$}
\begin{figure}
	\includegraphics[width=1.05\linewidth]{figures/fhvae_encoder_0.pdf}
\end{figure}
\end{frame} 
\setbeamercolor{background canvas}{bg=white}
\setbeamercolor{normal text}{fg=darkgrey}
\usebeamercolor[fg]{normal text}
\setbeamertemplate{itemize item}{\color{darkgrey}$\circ$}
\begin{frame}
\frametitle{FHVAE}
\framesubtitle{Decoder$_{\phi}$}
\begin{figure}
	\includegraphics[width=1.05\linewidth]{figures/fhvae_encoder_1.pdf}
\end{figure}
\end{frame} 
\setbeamercolor{background canvas}{bg=white}
\setbeamercolor{normal text}{fg=darkgrey}
\usebeamercolor[fg]{normal text}
\setbeamertemplate{itemize item}{\color{darkgrey}$\circ$}
\begin{frame}
\frametitle{FHVAE}
\framesubtitle{Decoder$_{\phi}$}
\begin{figure}
	\includegraphics[width=1.05\linewidth]{figures/fhvae_encoder_2.pdf}
\end{figure}
\end{frame} 





\setbeamercolor{background canvas}{bg=white}
\setbeamercolor{normal text}{fg=darkgrey}
\usebeamercolor[fg]{normal text}
\setbeamertemplate{itemize item}{\color{darkgrey}$\circ$}
\begin{frame}
\frametitle{FHVAE}
\framesubtitle{Objective}
%\begin{itemize}
%	\item regularize $z_1$ globally (through standard Gaussian prior)
%	\item regularize $z_2$ through sequence dependent prior $\tilde{\mu_2}$
%\end{itemize}
\begin{align*}
\mathcal{L}(\theta, \phi, X)& = \sum_{n=1}^N \color{sienna}\underbrace{\mathcal{L}(\theta, \phi;x^{(n)}|\mu_2) \color{black}+\color{black} log\;p_{\theta}(\mu_2)\color{black} + const.}_{\text{var. lower bound}} + \color{cornflower}\underbrace{\color{black}\alpha \cdot \color{cornflower}log\;(i|z_2^{(i,n)})}_{\text{discrim.obj.}}\\
\text{with }\color{sienna}\mathcal{L}(\theta, \phi;x^{(n)}|\mu_2) &\color{sienna}= \mathbb{E}_{q_{\phi}(z_1^{(n)},z_2^{(n)}|x^{(n)})}\left[ log\;p_{\theta}(x^{(n)}|z_1^{(n)},z_2^{(n)}) \right]\\
-& \color{sienna}\mathbb{E}_{q_{\phi}(z_2^{(n)}|x^{(n)})} \left[ D_{KL}(q_{\phi}(z_1^{(n)}|x^{(n)},z_2^{(n)}) ||\underbrace{p_{\theta}(z_1^{(n)})}_{\text{sequ. ind.}}) \right]\\
-&\color{sienna}D_{KL}(q_{\phi}(z_2^{(n)}|x^{(n)})||\underbrace{p_{\theta}(z_2^{(n)}|\mu_2)}_{\text{seq. dep. prior}})\\
\color{black}\text{and }\color{cornflower}log\;(i|z_2^{(i,n)}) &=\color{cornflower} log\;p(z_{2}^{(i,n)}|\mu_2^{(i)}) - log(\sum_{j=1}^{M}p(z_2^{(i,n)}|\mu_2^{(j)}))
\end{align*}
\end{frame} 


\setbeamercolor{background canvas}{bg=white}
\setbeamercolor{normal text}{fg=darkgrey}
\usebeamercolor[fg]{normal text}
\setbeamertemplate{itemize item}{\color{darkgrey}$\circ$}
\begin{frame}
\frametitle{FHVAE}
\framesubtitle{S-vector $\mu_2$}
\begin{itemize}
	\item What is sequence dependent prior $\mu_2$ (s-vector)?
	\begin{itemize}
		\item Imagine a word vector 
		\item S-vector for every sequence
		\item Similar sequence attributes $\rightarrow$ s-vectors close in euclidean space
		%\item Ideally, similarities in sequences should  result in s-vectors close in euclidian space
		\item $g(sequence\;id) = \mu_2$ as (differentiable) lookup table
		\begin{itemize}
			\item $\rightarrow$ Embedding in pytorch, tensorflow
		\end{itemize}
		%\item During test, s-vector can be found in closed form solution %where there is no seq.id available
	\end{itemize}
\end{itemize}
\end{frame} 



\setbeamercolor{background canvas}{bg=white}
\setbeamercolor{normal text}{fg=darkgrey}
\usebeamercolor[fg]{normal text}
\setbeamertemplate{itemize item}{\color{darkgrey}$\circ$}
\begin{frame}
\frametitle{FHVAE}
\framesubtitle{Objective - Sequence variational lower bound}
\begin{align*}
\mathcal{L}(\theta, \phi, X)& = %\sum_{n=1}^N \mathcal{L}(\theta, \phi;x^{(n)}|\tilde{\mu_2}) + log\;p_{\theta} + const.\\
%\mathcal{L}(\theta, \phi;x^{(n)}|\tilde{\mu_2}) &= 
\underbrace{log\;p(x|z_1, z_2)}_{\text{reconstruction}}\\%prob. of obs. data under learned posterior\\
-&\underbrace{D_{KL}(\mathcal{N}(\mu_{z_1}, \sigma_{z_1})||\mathcal{N}(0,1))}_{\text{regularize $z_1$ with global prior}}\\
-&\underbrace{D_{KL}(\mathcal{N}(\mu_{z_2}, \sigma_{z_2})||\mathcal{N}(\mu_2,0.5))}_{\text{regularize $z_2$ with seq. dep. prior $\mu_2$}}\\
\color{lightgray}+&\color{lightgray}\underbrace{log\;p(\mu_2) \cdot \frac{1}{seq.\;length}}_{\text{prob. of $\mu_2$ under standard Gaussian prior}}
\end{align*}
\end{frame} 



\setbeamercolor{background canvas}{bg=white}
\setbeamercolor{normal text}{fg=darkgrey}
\usebeamercolor[fg]{normal text}
\setbeamertemplate{itemize item}{\color{darkgrey}$\circ$}
\begin{frame}
\frametitle{FHVAE}
\framesubtitle{Objective - Discriminative objective}
%\begin{align*}
%log\;p(sequence\;id | z_2)& = CrossEntropy(\frac{-(\mu_{z_2} - \mu_2)^2}{\sigma_{z_2}^2},\;sequence\;id)
%\end{align*}
\begin{align*}
log\;p(sequence\;id^{(i)} | z_2^{(i,n)})& = log\;\frac{p(\mu_{z_2}^{(i,n)}|\mu_2^{(i)})}{\sum^{num\;seqs}_{j=1}	p(\mu_{z_2}^{(i,n)}|\mu_2^{(j)})}
\end{align*}
\begin{itemize}
	\item Encourage $z_2^{(i)}$ to be close to corresponding $\mu_2^{(i)}$
	\item and far from all other sequence's s-vectors $\mu_2^{(j \neq i)}$
%	\item $\rightarrow$ Try to predict sequence id with $z_2$
%	\item Encourage $\mu_2, z_2$ to become more expressive
\end{itemize}
\end{frame} 



\setbeamercolor{background canvas}{bg=white}
\setbeamercolor{normal text}{fg=darkgrey}
\usebeamercolor[fg]{normal text}
\setbeamertemplate{itemize item}{\color{darkgrey}$\circ$}
\begin{frame}
\frametitle{FHVAE}
\framesubtitle{Objective - Discriminative segment variational lower bound}
\begin{align*}
\mathcal{L}^{dis}(\theta, \phi;x)& = \mathcal{L}(\theta, \phi, X) + \alpha \cdot log\;p(sequence\;id | z_2)
\end{align*}
\begin{itemize}
	\item Joint objective to encourage factorization
	%\item discriminative objective can be adjusted through $\alpha$ hyperparameter
	\item $\alpha$ hyperparameter to weigh discriminative objective
\end{itemize}
\end{frame} 









\setbeamercolor{background canvas}{bg=white}
\setbeamercolor{normal text}{fg=darkgrey}
\usebeamercolor[fg]{normal text}
\setbeamertemplate{itemize item}{\color{darkgrey}$\circ$}
\begin{frame}
\frametitle{FHVAE}
\framesubtitle{Results - TIMIT Speaker Verification} % Maybe rather do ASR here
\begin{itemize}
	\item Task: Speaker verification
	\begin{itemize}
		\item Allows quantitative analysis of performance
		\item Assess quality of disentanglement
		\item Use s-vector $\mu_2$ to predict speaker
	\end{itemize}
	\item Compare i-vector baseline
	\begin{itemize}
		\item i-vector used in SOTA speaker verification approaches
		\item Low dimensional subspace of GMM universal background model
		\item Contains speaker information (content-independent)
		%\item contain sequence level info (-> speaker)
		%\item Sequence features from this GMM
	\end{itemize}
\end{itemize}
\end{frame} 


\setbeamercolor{background canvas}{bg=white}
\setbeamercolor{normal text}{fg=darkgrey}
\usebeamercolor[fg]{normal text}
\setbeamertemplate{itemize item}{\color{darkgrey}$\circ$}
\begin{frame}
\frametitle{FHVAE}
\framesubtitle{Results - TIMIT Speaker Verification} % Maybe rather do ASR here
\begin{itemize}
	\item Unsupervised speaker verification (Raw column)
	\item Metric: equal error rate (lower is better)
	\item $\mu_1$ based on $z_1$ as sanity check
\end{itemize}
\begin{figure}
	\includegraphics[width=.8\linewidth]{figures/timit_error.png}
	%\caption{Comparison of speaker verification equal error rate (EER) on TIMIT dataset (lower is better)}
	%\caption{World states are determinded by two generative factors, a change in those is reflected in the two corresponding latent variables, our $f$ maps to}
\end{figure}
\end{frame} 



\setbeamercolor{background canvas}{bg=white}
\setbeamercolor{normal text}{fg=darkgrey}
\usebeamercolor[fg]{normal text}
\setbeamertemplate{itemize item}{\color{darkgrey}$\circ$}
\begin{frame}
\frametitle{FHVAE}
\framesubtitle{Looking Back \& Discussion} % Maybe rather do ASR here
\begin{itemize}
	\item Evidence towards disentangling with respect to sequence-segment decomposition
	\begin{itemize}
		\item Other decompositions may prove more challenging
		%\item speaker gender, speaker age, language as more fine grained decompositions
	\end{itemize}
	\item Good performance on speaker verification and denoising task
	\item I had trouble disentangling simple examples
	\item Questions for you:
	\begin{itemize}
		\item Is learning disentangled representations worth the effort?
		\item Have there been situations where you wished for an interpretable latent space?
		\item Do you know any successful models where equivariant maps are used?
		%\item Are there any models you know, which use disentangled representations (with respect to some decomposition)
	\end{itemize}
\end{itemize}
\end{frame} 


\setbeamercolor{background canvas}{bg=white}
\setbeamercolor{normal text}{fg=darkgrey}
\usebeamercolor[fg]{normal text}
\setbeamertemplate{itemize item}{\color{darkgrey}$\circ$}
\begin{frame}
\frametitle{References}
\framesubtitle{}
\begin{itemize}
	\item Hsu, W.N., Zhang, Y. and Glass, J., 2017. Unsupervised learning of disentangled and interpretable representations from sequential data. In Advances in neural information processing systems (pp. 1878-1889).
	\item Hsu, W.N. and Glass, J., 2018. Scalable factorized hierarchical variational autoencoder training. arXiv preprint arXiv:1804.03201.
	\item Higgins, I., Amos, D., Pfau, D., Racaniere, S., Matthey, L., Rezende, D. and Lerchner, A., 2018. Towards a definition of disentangled representations. arXiv preprint arXiv:1812.02230.
	\item Scott, W.R., 2012. Group theory. Courier Corporation.
	\item Kingma, D.P. and Welling, M., 2013. Auto-encoding variational bayes. arXiv preprint arXiv:1312.6114.
\end{itemize}
\end{frame} 










\setbeamercolor{background canvas}{bg=white}
\setbeamercolor{normal text}{fg=darkgrey}
\usebeamercolor[fg]{normal text}
\setbeamertemplate{itemize item}{\color{darkgrey}$\circ$}
\begin{frame}
\frametitle{Challenges}
\framesubtitle{...}
\begin{itemize}%\setlength\itemsep{1.5em}
	\item If we really think about it, it is hard for us to define what a disentangled representation should actually be
	\item Precise biases of what the latent space should be decomposited into can be helpful as well as biases towards the 'form' of these latent subspaces
\end{itemize}
\end{frame} 

\end{document}
