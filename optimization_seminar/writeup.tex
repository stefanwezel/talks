\newcommand{\COURSE}{Seminar: OPT/NAS 20/21}
\newcommand{\STUDENT}{Stefan Wezel}

\documentclass[a4paper]{scrartcl}

\usepackage[utf8]{inputenc}
\usepackage[ngerman]{babel}
\usepackage{amsmath}
\usepackage{amssymb}
\usepackage{fancyhdr}
\usepackage{color}
\usepackage{graphicx}
\usepackage{lastpage}
\usepackage{listings}
\usepackage{tikz}
\usepackage{pdflscape}
\usepackage{subfigure}
\usepackage{float}
\usepackage{polynom}
\usepackage{hyperref}
\usepackage{tabularx}
\usepackage{forloop}
\usepackage{geometry}
\usepackage{listings}
\usepackage{fancybox}
\usepackage{tikz}
\usepackage{algpseudocode,algorithm,algorithmicx}

%Definiere Let-Command für algorithmen
\newcommand*\Let[2]{\State #1 $\gets$ #2}

\input kvmacros

%Größe der Ränder setzen
\geometry{a4paper,left=3cm, right=3cm, top=3cm, bottom=3cm}

%Kopf- und Fußzeile
\pagestyle {fancy}
\fancyhead[L]{\STUDENT}
\fancyhead[C]{\COURSE}
\fancyhead[R]{\today}

\fancyfoot[L]{}
\fancyfoot[C]{}
\fancyfoot[R]{Page \thepage /\pageref*{LastPage}}
\title{\textbf{Loss Landscape Visualization}\\\small Report for the Seminar on Optimization and Neural Architecture Search}
\begin{document}	
	
	
	% ----------------------- TODO ---------------------------
	% Hier werden die Aufgaben/Lösungen eingetragen:

%\author{Stefan Wezel}	
\maketitle
%\tableofcontents
\section*{}


	% TODO remove sample text

\section*{Abstract}
\textit{Neural networks have emerged as powerful function approximators with large parameter sets. These parameters are optimized according to a loss function. Many assumptions have been made about the shape of this loss landscape. However, only recently qualitative and empricial studies have been conducted.}

\section*{Introduction}

\section*{Related Work}

\section*{Methods}
\subsection*{Linear Interpolation}
\subsection*{Filter Normalization}

\section*{Results}


\section*{Conclusions}



	
\end{document}
%%% Local Variables:
%%% mode: latex
%%% TeX-master: t
%%% End: