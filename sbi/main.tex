\documentclass[12pt]{article}

% Some typesetting decisions
\usepackage[utf8]{inputenc}

% Typesetting and basic formatting. 
% This PDF is made to be read on-screen,
% so we use large fonts and little margin
\usepackage[margin=30mm]{geometry}
\usepackage{fourier}
\usepackage{setspace}
\setstretch{1.1}
\setlength{\parindent}{0pt}
\setlength{\parskip}{12pt}
\frenchspacing
\sloppy


% Links
\usepackage{xcolor}
\usepackage{hyperref}
\hypersetup{colorlinks,linkcolor={green!50!black},citecolor={blue!50!black}, urlcolor={red!50!black}}  
% This one is for demo-purposes only, don't use it in your submission ;) 
\usepackage{blindtext}

% Citation with natbib: 
% Use \citet{} and \citep{} only, not \cite{}
\usepackage{natbib}
\bibliographystyle{abbrvnat}



% Custom (maths) commands
\usepackage{amsmath, amssymb}
\usepackage{bm}

\newcommand{\Ebb}{\mathbb{E}}
\newcommand{\Rbb}{\mathbb{R}}
\newcommand{\Sbb}{{\mathbb{S}}}
\newcommand{\Tbb}{\mathbb{T}}
\newcommand{\Nbb}{\mathbb{N}}
\newcommand{\Xbb}{{\mathbb{X}}}
\newcommand{\Pbb}{\mathbb{P}}
\newcommand{\Dbb}{\mathbb{D}}
\newcommand{\Ybb}{\mathbb{Y}}
\newcommand{\onebb}{\mathbb{1}}
\newcommand{\Ncal}{\mathcal{N}}
\newcommand{\Mcal}{\mathcal{M}}
\newcommand{\Lcal}{\mathcal{L}}
\newcommand{\Bcal}{\mathcal{B}} 
\newcommand{\Qcal}{\mathcal{Q}} 
\newcommand{\Acal}{\mathcal{A}}
\newcommand{\Xcal}{\mathcal{X}}
\newcommand{\Tcal}{\mathcal{T}}
\newcommand{\Pcal}{\mathcal{P}}
\newcommand{\Zcal}{\mathcal{Z}}
\newcommand{\Rcal}{\mathcal{R}}
\newcommand{\Ocal}{\mathcal{O}}
\newcommand{\Fcal}{\mathcal{F}}
\newcommand{\Dcal}{\mathcal{D}}
\newcommand{\Ycal}{\mathcal{Y}}
\newcommand{\Ecal}{\mathcal{E}}
\newcommand{\Hcal}{\mathcal{H}}
\newcommand{\Ucal}{\mathcal{U}}
\newcommand{\Ccal}{\mathcal{C}}
\newcommand{\Ical}{\mathcal{I}}
\newcommand{\diff}{\,\text{d}}
\DeclareMathOperator{\diag}{diag}
\DeclareMathOperator{\cond}{cond}
\DeclareMathOperator{\gp}{GP}
\DeclareMathOperator{\divergence}{div}
\DeclareMathOperator{\curl}{curl}
\DeclareMathOperator{\supp}{supp}
\DeclareMathOperator*{\argmax}{arg\,max}
\DeclareMathOperator*{\argmin}{arg\,min}






% Insert the title of your presentation, your name, and your e-mail here.
% Leave the rest as is.
\title{\vskip-3em \bf 
	Simulation-based Inference
    }
\author{
    A Summary Written by Stefan Wezel \\
    \texttt{stefan.wezel@student.uni-tuebingen.de}
}
\date{\it Machine Learning for and with Dynamical Systems\\Summer Term 2021}


\begin{document}

\maketitle

\begin{abstract}\vskip-1.5em \noindent
    Summarise the main messages of the paper in 2-4 sentences. There is no reason to do more, because the entire document is super short anyway. Mention the work that your paper is based on, in this example:
    \citet{bosch2021calibrated}
\end{abstract}


\section*{Problem Setting}
\section*{Traditional Approximate Bayesian Computation}
\section*{Neural Approach}
\section*{Possible Enhancements}
\section*{Conclusion}




Write 2-3 pages that summarise your topic.

Use \texttt{citet}: \citet{bosch2021calibrated} and \texttt{citep}: \citep{bosch2021calibrated}
for proper referencing. Cite thoroughly. Check proper capitalisation in your references.

Use only the provided maths commands use \texttt{align} for embedded maths (not \texttt{align*}):
\begin{align}
    \Fcal, \Rbb, \Nbb
\end{align}
The \texttt{diff} command is for proper typesetting in integrals: $\int f(x) \diff x$, not $\int f(x) dx$.
If you need conditional probabilities, use \texttt{mid} for $A$ conditioned on $B$: $p(A \mid B)$.

Write clearly; pay attention to clarity, simplicity, and -- above all -- correctness of your statements. Look up how to write a scientific document.

\bibliography{references}
\end{document}